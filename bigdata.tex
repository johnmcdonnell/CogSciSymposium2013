% 
% Annual Cognitive Science Conference
% Sample LaTeX Paper -- Proceedings Format
% 

% Original : Ashwin Ram (ashwin@cc.gatech.edu)       04/01/1994
% Modified : Johanna Moore (jmoore@cs.pitt.edu)      03/17/1995
% Modified : David Noelle (noelle@ucsd.edu)          03/15/1996
% Modified : Pat Langley (langley@cs.stanford.edu)   01/26/1997
% Latex2e corrections by Ramin Charles Nakisa        01/28/1997 
% Modified : Tina Eliassi-Rad (eliassi@cs.wisc.edu)  01/31/1998
% Modified : Trisha Yannuzzi (trisha@ircs.upenn.edu) 12/28/1999 (in process)
% Modified : Mary Ellen Foster (M.E.Foster@ed.ac.uk) 12/11/2000
% Modified : Ken Forbus                              01/23/2004
% Modified : Eli M. Silk (esilk@pitt.edu)            05/24/2005
% Modified: Niels Taatgen (taatgen@cmu.edu) 10/24/2006

%% Change ``a4paper'' in the following line to ``letterpaper'' if you are
%% producing a letter-format document.

\documentclass[10pt,letterpaper]{article}

\usepackage{cogsci}
\usepackage{pslatex}
\usepackage{apacite}


\title{A Data Revolution in the Cognitive Sciences}
 
\author{{\large \bf John V. McDonnell (\emph{moderator}) and Todd M. Gureckis}  \\
  New York University Department of Psychology, 6 Washington Pl. 8th Floor \\
  New York, NY 10003 USA
  \AND {\large \bf Johan Bolen (jbollen@indiana.edu)} \\
  Indiana University School of Informatics and Computing 919 E 10th Street, Room 304\\
  Bloomington, IN 47408 USA
  \AND {\large \bf Georg Langs (georg.langs@meduniwien.ac.at)} \\
  Medical University of Vienna, Waehringer Guertel 18-20 \\
  A-1090 Vienna AUSTRIA
  \AND {\large \bf Winter Mason, Meeyoung Cha, Krishna Gummadi, Farshad Kooti, and Haeryun Yang} \\
  Howe School of Technology Managemeent, Stevens Institute \\
  Hoboken, NJ 07030, USA
  \AND {\large \bf John Myles White (jmw@johnmyleswhite.com)} \\
  Princeton University Department of Psychology \\
  Princeton, NJ 08540 USA}


\begin{document}

\maketitle


\textbf{Keywords:} 
Machine Learning; Cognitive Neuroscience; Online Experiments; Social Network
Analysis; Text Analysis; Methods\\

In the 1980s and 90s, the broad availability of PCs power fueled a renaissance in
cognitive science, changing the way we run experiments, the tools we use for
analysis, and lowering computational barriers to the development of sophisticated
computational models of cognition. As technology continues to advance, the advent
of big data is now leading to methodological advances on various fronts, from
online experimentation to the mining of large datasets. These new opportunities
have also come with new challenges, as scientists must overcome the challenges
these methods bring. The goal of this symposium would be to highlight how
researchers are making use of these new tools and the techniques they have used
to solve the challenges they represent. 

A major theme of the symposium will be the way in which these new techniques
represent more than just a scaling up of previous efforts: Researchers are now
able to ask new questions in different ways than they were before. For example,
collective behavior has long been of interest to cognitive scientists as a way of
understanding how human interactions lead to emergent group behavior and
collective intelligence. But until recently, it has been difficult to study
authentic human social interaction as it occurs in the environment on a large
scale. Now that scientists have access to anonymized data from massive social
networks, these interactions can be studied in the environment. Johan Bolen and
Winter Mason will be discussing work they have done using this newly available
data source.

Another novel source of data are online labor marketplaces such as Amazon's
Mechanical Turk (AMT) service. These allow researchers to perform controlled
studies involving large numbers of participants in short periods of time. This
allows researchers to run experiments requiring far more data than in the past,
such as brief one-shot studies with thousands of participants or learning task
optimization. John Myles White and John McDonnell will talk about their work
running studies online with AMT.

Cognitive neuroscience has also been a major beneficiary of the data revolution.
The prevalence of fMRI has led to the amassing of petabytes of scanning data.
Analysis of these data poses challenges in terms of statistical rigor (given the
problem of multiple comparisons on a massive scale) and managing immense
quantities of data. Georg Langs will speak about his work using cutting edge
machine learning techniques to derive new insights from imaging data.

\section{Machine Learning Approaches in Neuroimaging and Medical Image Analysis.}
\subsubsection{Georg Langs}

The computational analysis of biomedical imaging data has become central in
medical research and clinical application. The talk will highlight two recent
developments that might shape research in this area in the near future. Both are
connected to the impact of the emerging ability to process large data, and the
approaches that become relevant. The advent of the capability to process large
data---hundreds of terabytes---has changed paradigms in how we approach classical
pattern recognition problems in this domain. Medical research fields that rely on
exploratory approaches together with large and complex data to study and
understand physiological processes are increasingly shaped by a tight interaction
among researchers in machine learning and medical sciences. In the talk I will
discuss these directions, highlight open issues, and illustrate them
with examples from medical image retrieval, and neuroimaging research and
clinical application.

\section{The emergence of social conventions in social networks}
\subsubsection{Winter Mason, Meeyoung Cha, Krishna Gummadi, Farshad Kooti, and Haeryun Yang}

Social conventions and norms are a powerful and ubiquitous guide for behavior.
However, the way in which conventions emerge in communities is not very well
understood, largely due to a paucity of available data.  In this study, we
leverage the widespread use of the micro-blogging platform Twitter and focus on
competing conventions for attributing reposts to the original source.  We analyze
over 1.7 billion ``tweets'' from 54 million users, from the very first tweet in
2006 up to September 2009, and observe how the conventions emerged and spread
through the network of Twitter users.  We observe that initially the most
successful conventions were borrowed from natural language (``via'' and
``retweeting''), but eventually a community-specific convention came to dominate
(``RT'').  We see that some conventions are used by divergent groups of people,
while others are abandoned in favor of more efficient expressions.  We also
observe the failure of some conventions despite higher efficiency (i.e., fewer
characters) and explicit endorsement of their adoption.  Our results suggest that
there are some features that encourage the adoption of one convention over
another, but that there is still significant inherent unpredictability in what
convention will come to dominate.

\section{Context and decision making in a massive online experiment}
\subsubsection{John Myles White}

Online labor markets, like Amazon's Mechanical Turk (AMT), offer psychologists
many opportunities.  The convenience of the virtual lab provided by AMT has
already won over many psychologists, but transitioning research from the lab to
the web browser offers other benefits. AMT allows psychologists to deploy
experiments that are fully automated: The recruitment, instruction, core
experiment and debriefing periods can be identical for all subjects, which
largely removes the possibility that undocumented components of an experiment
might exert substantive influences on the final results (Doyen et al., 2012).
\nocite{Doyen} Moreover, the ease of recruiting large numbers of subjects for
tasks provides important increases in statistical power, which can allay concerns
about false positive psychology (Simmons et al., 2012).  \nocite{Simmons} But I
will argue that the primary value of AMT is not purely methodological: the
strength of the virtual lab is that it allows psychologists to pursue large-scale
between-subjects designs in which a large number of subjects perform a very small
number of trials.  This type of research is often eschewed because of the
prohibitive cost of recruiting hundreds or thousands of subjects in the lab, but
previous research (e.g. Gneezy et al. 2006) \nocite{Gneezy} suggests that studies
of decision-making can be powerfully influenced by contextual cues.  Our recent
work has found evidence that the effects of context on decisions may be even more
problematic than previously believed: in our studies of decision-making, we find
that the effects of ostensibly innocuous local context can shift the basic
qualitative results of experiments.

\section{Can online data be trusted? Learning tasks on Amazon's Mechanical Turk.}
\subsubsection{John V. McDonnell, Todd M. Gureckis}

Amazon Mechanical Turk (AMT) has attracted attention from experimental
psychologists interested in gathering human subject data more efficiently.
However, relative to traditional laboratory studies, many aspects of the testing
environment are not under the experimenter's control. We have empirically
evaluated the fidelity of AMT for use in cognitive behavioral experiments. These
types of experiment differ from simple surveys in that they require multiple
trials with sustained attention from participants.  Our initial attempts to
replicate the classic Shepard, Hovland, and Jenkins (1961) task were only
partially successful. However, after systematically studying the effects of
compensation and validation we were able to more closely match previous in-lab
findings. Specifically, we found that compensation altered the rate of sign-ups,
but not the quality of the data. Conversely, we found that using a strict measure
to ensure that participants had understood instructions resulted in a
considerable improvement in performance and convergence with in-lab findings.
\nocite{Shepard:1961wo}




\bibliographystyle{apacite}

\setlength{\bibleftmargin}{.125in}
\setlength{\bibindent}{-\bibleftmargin}

\bibliography{symposium}


\end{document}
